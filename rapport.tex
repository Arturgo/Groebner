\documentclass{article}
\usepackage[utf8]{inputenc}
\usepackage{amsmath}
\usepackage{amsfonts}
\usepackage{dsfont}
\usepackage[french]{babel}
\usepackage{tikz-cd}
\usepackage{amsthm}
\usepackage{amssymb}
\usepackage{wrapfig}
\usepackage{accents}
\usepackage{url}
\usepackage[backend=biber]{biblatex}
\usepackage[ruled,vlined]{algorithm2e}
\bibliography{rapport}

\title{Algorithmes à signature pour le calcul des bases de Gröbner non-commutatives}
\author{Arthur Léonard}
\date{2021}

\newtheorem*{thm}{Théorème}
\newtheorem*{lem}{Lemme}
\newtheorem*{fact}{Fait}
\newtheorem*{dfn}{Définition}
\newtheorem*{cor}{Corollaire}

\newcommand{\N}{\mathbb{N}}
\newcommand{\K}{\mathbb{K}}
\newcommand{\chev}[1]{\langle #1 \rangle}
\newcommand{\A}{\mathbb{A}}
\newcommand{\M}{\mathcal{M}}
\newcommand{\CR}{\K[\M\times\{1, \dots, m\}\times\M]}

\newcommand{\Ind}{\mathds{1}}
\DeclareMathOperator{\LM}{LM}
\DeclareMathOperator{\LC}{LC}
\DeclareMathOperator{\LT}{LT}
\DeclareMathOperator{\Sig}{Sig}
\DeclareMathOperator{\supp}{supp}
\SetKw{Break}{break}
\SetKw{Continue}{continue}
\SetKwInOut{Input}{Entrée}
\SetKwInOut{Output}{Résultat}

\begin{document}

\maketitle

\section*{Introduction}
Dans les années 1960, Bruno Buchberger introduit les bases de Gröbner, et trouve un algorithme qui permet de les calculer \cite{buchberger}. Cependant, cet algorithme, basé sur des divisions polynomiales multivariées répétées, effectue de nombreuses divisions dont le reste est nul, qui ne permettent pas de découvrir des nouveaux polynômes dans la base. Pour améliorer significativement la performance des algorithmes de calcul des bases de Gröbner, de nombreux critères ont été trouvés pour éviter d'effectuer ces divisions.

En 1999 et 2002, Jean-Charles Faugère introduit deux algorithmes, F4 \cite{f4} et F5 \cite{f5}. Le premier utilise de l'algèbre linéaire pour effectuer plusieurs réductions en parallèle. En revanche, le second utilise l'information donnée par les divisions à zéro, en étudiant le module des syzygies. C'est le premier algorithme à signature. L'algorithme F5 a ensuite été simplifié par Gao, Guan et Volny dans l'algorithme G2V \cite{g2v}, qui a lui-même été amélioré et généralisé par Gao, Volny et Wang en 2010 dans l'algorithme GVW.

Dans le mémoire, nous avons présenté l'algorithme GVW \cite{gvw}, dans le cas particulier d'un anneau de polynômes sur un corps, avec un seul ordre monomial et l'ordre POT (Position Over Term).

L'objectif du stage, et du rapport qui suit, est d'étudier comment les algorithmes à signature peuvent se généraliser au contexte non-commutatif. Nous avons adapté de nombreuses définitions en non-commutatif, et nous avons remarqué à fortiori que nos définitions coïncidaient avec celles d'un autre article \cite{noncom}. Une fois ces définitions posées, nous nous sommes heurtés à des difficultés que nous avons essayé d'identifier au mieux. Nous avons ainsi démontré un résultat qui montre une obstruction à la création d'un algorithme à signature non-commutatif : l'ensemble des monômes dominants des syzygies, ensemble essentiel à la création d'un algorithme à signature, n'est pas nécessairement un langage décidable. En ajoutant pour contrainte que l'ordre soit isomorphe à $\N$ pour les ordres ("fair" dans \cite{noncom}), nous avons réussi à démontré que l'ensemble des monômes dominants des syzygies n'est pas algébrique.

\section*{Algorithme de Buchberger non-commutatif}
Commençons par adapter les différentes définitions du mémoire au cadre non-commutatif.
Soit $\K$ un corps, on se place dans $\A = \K \chev{X_1, \dots, X_n}$, l'ensemble des polynômes non-commutatifs à $n$ variables à coefficients dans $\K$. On note~: 
$$\M := \{X_{i_1}X_{i_2} \dots X_{i_k} \;|\; k \in \N, (i_1, i_2, \dots, i_k) \in \{1, \dots, n\}^k \}$$
l'ensemble des monômes non-commutatifs. On fixe un ordre monomial $\leq$, c'est à dire un bon ordre sur $\M$ qui vérifie~:
\begin{itemize}
\item $\forall A \in \M, 1 \leq A$
\item $\forall A, B \in \M, (A \leq B \implies \forall M \in \M, AM \leq BM)$
\item $\forall A, B \in \M, (A \leq B \implies \forall M \in \M, MA \leq MB)$
\end{itemize}
Pour $f = \sum_{M \in \M} \alpha_M M \in A$, on notera~:
\begin{itemize}
\item $\LM(f) := \sup \{ M \in \M \;|\; \alpha_M \neq 0 \}$ le monôme dominant de $f$.
\item $\LC(f) := \alpha_{\LM(f)}$ le coefficient dominant de $f$.
\item $\LT(f) := \LC(f)\LM(f)$ le terme dominant de $f$.
\end{itemize}
Soit $|$ la relation de divisibilité sur $\M$, définie par~:
$$\forall A, B \in \M, (A \;|\; B \iff \exists L, R \in \M, LAR = B)$$

\begin{dfn}[Base de Gröbner]
	Soit $I$ un idéal de $\A$, et soit $G$ un ensemble de polynômes de $I$.
	Si pour tout $f \in I$, il existe $g \in G$ tel que $\LM(g) \;|\; \LM(f)$, on dit que $G$ est une base de Gröbner de $I$.
\end{dfn}

\begin{fact}
	Contrairement au cas commutatif, tous les idéaux ne sont pas finiment engendrés, et certains idéaux finiment engendrés n'ont pas de base de Gröbner finie.
\end{fact}
\begin{proof}
	Si l'on savait trouver une base de Gröbner finie de tout idéal finiment engendré, on pourrait décider le problème du mot en calculant une base de Gröbner de $(A_i - B_i)_{i \in \{1, \dots, k\}}$ qui correspondent aux règles de réécriture $A_i \rightarrow B_i$. 
	Or le problème du mot est indécidable.
\end{proof}

Soit $I$ un idéal de type fini, engendré par $g_1, \dots, g_m$, nous disposons de l'algorithme de Buchberger non-commutatif qui nous permet de calculer une base de Gröbner finie de $I$ si elle existe. 

L'algorithme de Buchberger non-commutatif est en fait très similaire à l'algorithme commutatif, à l'exception que les S-polynômes de deux polynômes $P$ et $Q$ sont de la forme $lPr - l'Qr'$, où $l, r, l', r \in \M$ sont tels que $l\LM(P)r = l'\LM(Q)r'$, $\deg{l\LM(P)} > \deg{l'}$ et $\deg{\LM(Q)r'} > \deg{r}$.

Cependant, comme dans le cas commutatif, cet algorithme effectue des réductions à zéro que nous voudrions éviter.

\subsection*{Exemple d'exécution}

Considérons $g_1 = ts - a, g_2 = sa - at$ avec l'ordre deglex.
On obtient dans l'ordre les S-polynômes suivants~:
\begin{itemize}
\item $g_1 a - t g_2 = tat - aa$, qui est irréductible, on ajoute donc $g_3 := tat - aa$ à la base de Gröbner.
\item $g_3 s - ta g_1 = taa - aas$, qui est irréductible, on ajoute donc $g_4 := taa - aas$ à la base de Gröbner.
\item $ta g_3 - g_3 at = -taaas + aaaa$, qui se réduit de la manière suivante~:
$$-taaas + aaaa \rightarrow -aasas + aaaa \rightarrow -aaats + aaaa \rightarrow 0$$
On fait donc une première réduction à $0$.
\item $ta g_3 - g_3 at = -taaa + aaat$, qui se réduit de la manière suivante~:
$$-taaa + aaat \rightarrow -aasa + aaat \rightarrow 0$$
On fait donc une deuxième réduction à $0$.
\end{itemize}
La base de Gröbner obtenue est $(ts - a, sa - at, tat - aa, taa - aas)$.

Comme il était fastidieux d'exécuter l'algorithme de Buchberger à la main sur des exemples, ce dernier a été implémenté.

\section*{Signatures}

Nous aimerions donc adapter le concept de signature au cadre non-commutatif, afin d'obtenir un algorithme qui permette de calculer une base de Gröbner finie de $I$ si elle existe, avec peu de réductions à zéro, grâce à un théorème F5 non-commutatif.

Comme nous allons le voir, les concepts s'adaptent naturellement, mais qu'il est difficile de garantir la terminaison d'un algorithme à signature.

\begin{dfn}[Chemins de réduction]
	On appelle $\CR$ l'ensemble des chemins de réduction.
	Cet ensemble est muni d'une structure naturelle de $\A$-bimodule et d'un morphisme 
	\begin{align*}
		E : \CR &\rightarrow \A \\
		\sum_{(l, i, r) \in \M\times\{1,\dots,m\}\times\M} \alpha_{(l,i,r)}(l,i,r) &\mapsto \sum_{(l, i, r) \in \M\times\{1,\dots,m\}\times\M} \alpha_{(l,i,r)}lg_ir
	\end{align*}
\end{dfn}

Soit $\leq$ un ordre sur $\M\times\{1,\dots,m\}\times\M$ compatible avec la multiplication à gauche et à droite par un élément de $\M$.
Pour 
$$\sigma = \sum_{(l, i, r) \in \M\times\{1,\dots,m\}\times\M} \alpha_{(l,i,r)}(l,i,r) \in \CR,$$
on notera~:
\begin{itemize}
\item $\LM(\sigma) := \sup \{ (l, i, r) \in \M\times\{1,\dots,m\}\times\M \;|\; \alpha_{(l, i, r)} \neq 0 \}$ le monôme dominant de $\sigma$.
\item $\LC(\sigma) := \alpha_{\LM(\sigma)}$ le coefficient dominant de $\sigma$.
\item $\LT(\sigma) := \LC(\sigma)\LM(\sigma)$ le terme dominant de $\sigma$.
\end{itemize}

\begin{dfn}[Sigpolynômes]
	On note :
	$$M := \{(\sigma, f) \in \CR\times\A \;|\; E(\sigma) = f\}$$
	Les élements de $M$ sont appelés sigpolynômes.
	Les premières composantes des élements de $M$ de second membre nul sont appelés syzygies.
	
	Pour $p = (\sigma, f) \in \CR\times\A$, on notera :
	\begin{itemize}
	\item $\LM(p) := \LM(f)$ le monôme dominant de $p$.
	\item $\Sig(p) := \LM(\sigma)$ la signature de $p$.
	\end{itemize}
\end{dfn}

\begin{dfn}[Réduction en tête]
	On dit que le sigpolynôme $(\sigma, f)$ est réductible en tête par le sigpolynôme $(\sigma', f')$ si on est dans l'un des cas suivants :
	\begin{itemize}
	\item $f' = 0$ et il existe $l, r \in \M$ tels que $\LM(\sigma) = l\LM(\sigma')r$.
	\item Il existe $l, r \in \M$ tels que $\LM(f) = l\LM(f')r$ et $l\LM(\sigma')r \leq \LM(\sigma)$.
	\end{itemize}
\end{dfn}

Les algorithmes à signatures calculent tous une base de Gröbner forte :

\begin{dfn}[Base de Gröbner forte]
	Une base de Gröbner forte de $M$ est un sous-ensemble $G_f$ de $M$ tel que tout élément non-nul de $M$ est réductible en tête par un élément de $G_f$.
\end{dfn}

L'intérêt de regarder une base de Gröbner forte est le théorème suivant~:

\begin{thm}[Théorème de projection]
	Soit $G_f$ une base de Gröbner forte de $M$, alors :
	$$\pi_2(G_f) := \{v \;|\; (u, v) \in M\}$$
	est une base de Gröbner de $I$.
\end{thm}
\begin{proof}
	Soit $f \in I$ non-nul. Par définition, il existe donc $\sigma \in \CR$ tel que $E(\sigma) = f$, d'où $(\sigma, f) \in M$. Choisissons un $\sigma$ tel que $\LM(\sigma)$ soit minimal.
	Comme $G_f$ est une base de Gröbner forte de $M$, il existe $(\sigma', f') \in G_f$ tel que $(\sigma, f)$ est réductible en tête par $(\sigma', f')$.
	Si $f' = 0$, alors il existe $l, r \in \M$ tel que $\LM(\sigma) = l\LM(\sigma')r$, et alors :
	$$(\sigma - \LC(\sigma)l\LM(\sigma')r, f) \in M,$$
	ce qui contredit la minimalité de $\LM(\sigma)$.
	
	Si $f' \neq 0$, alors il existe $l, r \in \M$ tel que $\LM(f) = l\LM(f')r$, d'où $\LM(f') \;|\; \LM(f)$ et $\pi_2(G_f)$ est bien une base de Gröbner de $I$.
\end{proof}

L'avantage (mais aussi l'inconvénient) d'un algorithme qui calcule une base de Gröbner forte est qu'il calcule aussi une base de Gröbner des syzygies.

\begin{dfn}[Base de Gröbner des syzygies]
	On dit qu'un sous-ensemble $G_s$ des syzygies est une base de Gröbner des syzygies, si pour chaque syzygie $\sigma$ non-nulle, il existe $\sigma' \in G_s$ et $l, r \in \M$ tels que $\LM(\sigma) = l\LM(\sigma')r$.
\end{dfn}

\begin{fact}
Lorsque l'on considère les premières composantes des sigpolynômes de seconde composante nulle d'une base de Gröbner forte, on obtient une base de Gröbner des syzygies.
\end{fact}

Il se trouve que l'on connaît déjà de nombreuses syzygies :

\begin{dfn}[Syzygies principales]
	On appelle syzygie principale une syzygie qui est dans le sous-$\A$-bimodule de $\CR$ engendré par les :
	$$g_i(X, j, 1) - (1, i, X)g_j, X \in \M, (i, j) \in \{1, \dots, m\}^2$$
	On dira que la suite $(g_1, \dots, g_m)$ est régulière s'il n'y a pas d'autres syzygies que les syzygies principales.
	
\end{dfn}

Par exemple, prenons $g_1 = ba - ab, g_2 = b^3$, avec pour ordres deglex et poslexdeglex.
Les générateurs des syzygies principales sont de la forme :
\begin{align*}
& (1, 2, Xba) - (1, 2, Xab) - (b^3X, 1, 1) \\
& (baX, 2, 1) - (abX, 2, 1) - (1, 1, Xb^3) \\
& (b^3X, 2, 1) - (1, 2, Xb^3) \\
& (baX, 1, 1) - (abX, 1, 1) - (1, 1, Xba) + (1, 1, Xab) \\
\end{align*}
Pour n'importe quel $X \in \M$.

Remarquons que l'on peut construire d'autres syzygies principales à partir des ces générateurs :
Par exemple, si on multiplie le deuxième gérateur par $b^2$ à gauche et qu'on fait le remplacement $X \rightarrow aX$ dans la troisième relation, on obtient~:
$$(b^2abX, 2, 1) - (1, 2, aXb^3) + (b^2, 1, Xb^3) $$
En simplifiant par la deuxième relation, on obtient :
$$(bab^2X, 2, 1) - (1, 2, aXb^3) + (b^2, 1, Xb^3) + (b, 1, bXb^3)$$
Et en resimplifiant par la deuxième relation, on obtient :
$$(ab^3X, 2, 1) - (1, 2, aXb^3) + (b^2, 1, Xb^3) + (b, 1, bXb^3) + (1, 1, b^2Xb^3)$$
On peut alors simplifier par la troisième relation, on obtient :
$$(a, 2, Xb^3) - (1, 2, aXb^3) + (b^2, 1, Xb^3) + (b, 1, bXb^3) + (1, 1, b^2Xb^3)$$
Qui donne un nouveau type de monôme dominant. Cette approche peut-être systématisée en un algorithme à la Knuth-Bendix.

Un algorithme à signature "naïf" ferait exactement la même chose que GVW dans le nouveau langage :
il commence avec les paires $\{((1, i, 1), g_i) \;|\; 1 \leq i \leq m\}$, puis à chaque étape choisit la paire critique de plus petite signature, la réduit régulièrement par les paires déjà ajoutées, puis l'ajoute aux paires existantes, en ignorant éventuellement des paires dont la signature est un monôme dominant de syzygie principale. Ce dernier algorithme a été implémenté dans une version simple, ce qui a permis d'observer qu'il terminait très rarement.
En effet, un algorithme à signature déterminerait donc d'une manière ou d'une autre les monômes dominants des syzygies. Dans le cas commutatif, cette base est finie ; c'est une conséquence du lemme de Dickson.
En revanche, dans le cas non-commutatif, les syzygies ne sont pas forcément de type fini, ce qui nous empêche de faire cet algorithme naïf. Pour que notre algorithme ait une chance de terminer, il faut réussir à traiter les syzygies de manière différente, ou au moins les syzygies princiales si on veut qu'il termine dans le cas d'une suite régulière.

\section*{Exemples de calcul de syzygies}

\subsubsection*{Ordre TOP}

Considérons l'exemple $g_1 = b^2, g_2 = ba - ab$, avec pour ordre deglex d'un côté, et avec pour ordres deglex et l'ordre de concaténation, où
nous avons $(l, i, r) \leq (l', j, r')$ si l'une des conditions ci-desous est vérifiée~:
\begin{itemize}
	\item $l\LM(g_i)r < l'\LM(g_j)r'$
	\item $l\LM(g_i)r = l'\LM(g_j)r'$ et $\deg{l} \leq \deg{l'}$
	\item $l\LM(g_i)r = l'\LM(g_j)r'$ et $\deg{l} = \deg{l'}$ et $i \leq j$.
\end{itemize}

Nous cherchons à déterminer les monômes dominants des syzygies.
\begin{fact}
	Les monômes dominants des syzygies sont les paires $(l, i, r)$ dont $l$ contient au moins un $b$.
\end{fact}
\begin{proof}
	Commençons par montrer que tous les monômes dominants des syzygies contiennent un $b$ à gauche.
	
	Soit $\sigma$ une syzygie non-nulle.
	On construit un graphe $G_\sigma$ de la manière suivante~:
	\begin{itemize}
		\item Les sommets de $G_\sigma$ sont les 
		$$\{l(ba)r \;|\; (l, 2, r) \in \supp(\sigma)\} \cup \{l(ab)r \;|\; (l, 2, r) \in \supp(\sigma)\} \cup \{l(bb)r \;|\; (l, 1, r) \in \supp(\sigma)\}$$
		
		\item Pour chaque terme $\alpha(l, 2, r)$ de $\sigma$, on ajoute un arc entre $l(ba)r$ et $l(ab)r$ de pondération $\alpha$.
		\item Chaque sommet $w$ est étiqueté 
		$$\sum_{\alpha (l, 1, r) \mbox{ terme de } \sigma, w = l(b^2)r} \alpha$$
		\item Le fait que $\sigma$ est une syzygie implique que pour chaque sommet $v$, la somme de la pondération de $v$, de la pondération des arcs qui sortent de $v$, moins la pondération des arcs qui entre dans $v$ est nulle. Appellons cette propriété "équation d'équilibre" en $v$.
	\end{itemize}
	
	Soit $s$ le sommet de $G_\sigma$ le plus grand pour l'ordre monomial, par propriétés de l'ordre de concaténation, il n'y a pas d'arc qui entre dans $s$.
	
	Distinguons deux cas~:
	\begin{itemize}
	\item Le monôme dominant de $\sigma$ est de la forme $(l, 1, r)$. Dans ce cas, puisqu'on a choisit l'ordre de concaténation, on a $s = l(b^2)r$. Par équation d'équilibre en $s$, on obtient qu'il existe un monôme $(l', j, r')$ de $\sigma$ tel que $l'\LM(g_j)r' = l(b^2)r$.
	Si $j = 1$, alors $l'(b^2)r' = l(b^2)r$, et comme $\deg{l'} < \deg{l}$, on obtient que $l$ doit contenir un $b$.
	Si $j = 2$, alors $l'(ba)r' = l(b^2)r$, et comme $\deg{l'} \leq \deg{l}$, on obtient que $l$ doit contenir un $b$.
	
	\item Le monôme dominant de $\sigma$ est de la forme $(l, 2, r)$. Dans ce cas, puisqu'on a choisit l'ordre de concaténation, on a $s = l(ba)r$. Par équation d'équilibre en $s$, on obtient qu'il existe un monôme $(l', j, r')$ de $\sigma$ tel que $l'\LM(g_j)r' = l(ba)r$.
	Si $j = 1$, alors $l'(b^2)r' = l(ba)r$, et comme $\deg{l'} \leq \deg{l}$, on obtient que $l$ doit contenir un $b$.
	Si $j = 2$, alors $l'(ba)r' = l(ba)r$, et comme $\deg{l'} < \deg{l}$, on obtient que $l$ doit contenir un $b$.
	\end{itemize}
	
	Réciproquement, montrons que tous les monômes contenant un $b$ à gauche sont des monômes dominants de syzygies.
	Par compatibilité à la multiplication à gauche et à droite, il suffit de montrer que~:
	$$(ba^n, 1, 1) \mbox{ et } (ba^n, 2, 1)$$
	sont des monômes dominants des syzygies, ce qui est le cas car~:
	$$(ba^n, 1, 1) - (1, 2, a^{n - 1}b^2) - (aba^{n - 1}, 1, 1)$$
	est une syzygie de monôme dominant $(ba^n, 1, 1)$ et
	$$(ba^n, 2, 1) + \sum_{k = 0}^{n} (a^k, 2, a^{n - k}b) + (a^{n+1}, 1, 1) - \sum_{k = 0}^{n - 1} (a^k, 2, a^{n - 1 - k}ba) - (a^n, 1, a)$$
	est une syzygie de monôme dominant $(ba^n, 2, 1)$.	
\end{proof}

Ce fait implique qu'il n'y a pas de base de Gröbner forte finie de $(g_1, g_2)$.

\subsubsection*{Ordre POT}

Considérons maintenant l'exemple $g_1 = a^2, g_2 = ba - ab$, avec pour ordre deglex d'un côté, et avec pour ordres deglex et l'ordre POT, où
nous avons $(l, i, r) \leq (l', j, r')$ si l'une des conditions ci-desous est vérifiée~:
\begin{itemize}
	\item $i < j$
	\item $i = j$ et $\deg{l} \leq \deg{l'}$
	\item $i = j$ et $\deg{l} = \deg{l'}$ et $l\LM(g_i)r < l'\LM(g_j)r'$
\end{itemize}

Soit $\LM_s$ l'ensemble des monômes dominants des syzygies, on cherche à déterminer $\LM_s$.
\begin{fact}
	On a~:
	$$\LM_s \cap \{(l, 2, 1) \;|\; l \in \M\} = \{(l, 2, 1) \;|\; l \in \M, l \mbox{ contient un } a\}$$
\end{fact}
\begin{proof}
	Soit $\sigma$ une syzygie non-nulle, dont le monôme dominant est $(l, 2, 1)$.
	Par équation d'équilibre en $l(ba)$, on obtient qu'il doit exister un monôme $(l', j, r')$ de $\sigma$ qui doit vérifier l'une des conditions suivantes~:
	\begin{itemize}
	\item $j = 2$ et $l'(ba)r' = l(ba)$, comme $\deg{l'} < \deg{l}$, on obtient que $l$ doit contenir un $a$.
	\item $j = 2$ et $l'(ab)r' = l(ba)$, comme $\deg{l'} < \deg{l}$, on obtient que $l$ doit contenir un $a$.
	\item $j = 1$ et $l'(aa)r' = l(ba)$, dans ce cas, il est clair que $l$ doit contenir au moins un $a$.
	\end{itemize}
	
	Montrons maitenant que tout élément de la forme $(l, 2, 1)$ avec $l$ contenant au moins un $a$ est un monôme dominant de syzygie.
	Par compatibilité à la multiplication à gauche et à droite, il suffit de montrer que les~:
	$$(ab^n, 2, 1)$$
	sont des monômes dominants des syzygies, ce qui est le cas car~:
	$$(b^{n + 1}, 1, 1) - \sum_{k = 0}^n (b^{n - k}, 2, b^k a)
	- \sum_{k = 0}^n (ab^{n - k}, 2, b^k) - (1, 1, b^{n + 1})$$
	est une syzygie de monôme dominant $(ab^n, 2, 1)$.	
\end{proof}

Ce fait implique qu'il n'y a pas de base de Gröbner forte finie de $(g_1, g_2)$.

\section*{Langage régulier ?}
Au vu de la forme des syzygies principales, et de l'algorithme pour en calculer de nouvelles, on peut se demander si un automate peut reconnaître les monômes dominants des syzygies, lorsque la base de Gröbner est finie.
Nous allons démontrer que ce n'est pas le cas, en utilisant le fait que le langage $\{a^nb^n \;|\; n \in \N\}$ n'est pas régulier.
Considérons les polynômes~:
\begin{align*}
g_1 & := sa - tcx \\
g_2 & := xa - ax \\
g_3 & := xd - dx \\
g_4 & := ya - ay \\
g_5 & := yd - dy \\
g_6 & := xbb - dyb \\
g_7 & := cya - ccx \\
g_8 & := xbe - dze \\
g_9 & := dz - zb \\
g_{10} & := cz - za \\
g_{11} & := tz - s
\end{align*}
avec pour ordre $m_1 < m_2$ si~:
\begin{itemize}
\item Il y a moins de symboles $\infty$ dans $m_1$ que dans $m_2$.
\item Il y a autant de symboles $\infty$ dans $m_1$ que dans $m_2$, mais il y a moins de symboles $s$ dans $m_1$ que dans $m_2$.
\item Il y a autant de symboles $\infty$ et de symboles $s$ dans $m_1$ que dans $m_2$, mais
$m_1 <_{deglex} m_2$ où $a < b < c < d < x < y < z < e < t < s < \infty$.
\end{itemize}

Nous avons $(l, i, r) \leq (l', j, r')$ si l'une des conditions ci-desous est vérifiée~:
\begin{itemize}
	\item $l\LM(g_i)r < l'\LM(g_j)r'$
	\item $l\LM(g_i)r = l'\LM(g_j)r'$ et $i < j$
	\item $l\LM(g_i)r = l'\LM(g_j)r'$ et $i = j$ et $\deg{l} < \deg{l'}$.
\end{itemize}

L'idée est que les deux symboles $s, t, e$ sont présents aux extrémités du mot pour pouvoir faire des opérations sur le premier $a$ ou sur le dernier $b$.
Le premier $a$ va se transformer en $c$ en générant un $x$, les autres $a$ vont absorber un $y$, et générer un $x$ en se tranformant en $c$.
Le dernier $b$ va absorber un $x$, générer un $z$ et se transformer en $d$, les autres $b$ vont absorber un $x$, et générer un $y$ en se transformant en $d$.
Les $x$ et les $y$ sont des "messages" qui commutent avec les $a$ et les $d$. Un $x$ "dit" à la frontière entre les $b$ et les $d$ qu'un $a$ s'est transformé en $c$, alors qu'un $y$ "dit" à la frontière entre les $a$ et les $c$ qu'un $b$ s'est transformé en $d$, l'objectif étant que pour comparer le nombre de $a$ et de $b$, il y ait autant de transformations de chaque type.
Le $z$ est le message qui permet de retransformer les $d$ en $b$ et les $c$ en $a$ à la fin du processus. On reparlera du rôle de $\infty$ plus tard.

Pour mieux comprendre, prenons $n = 2$. On a que $(1, 11, a^2b^2e)$ est un monôme dominant de syzygie car on a les réécritures suivantes~:
\begin{align*}
saabbe & =_{1} tcxabbe =_{2} tcaxbbe =_{6} tcadybe \\
& =_{5} tcaydbe =_{4} tcyadbe =_{7} tccxdbe \\
& =_{3} tccdxbe =_{8} tccddze =_{9} tccdzbe \\
& =_{9} tcczbbe =_{10} tczabbe =_{10} tzaabbe \\
& =_{11} saabbe
\end{align*}

\begin{fact}
	Les monômes dominants des syzygies de la suite de polynômes précédente ne forment pas un langage régulier.
\end{fact}
\begin{proof}
Nous allons montrer que l'ensemble des monômes dominants des syzygies de la forme $(1, 11, a^nb^me)$ vérifient $n = m$. Cela conclura, puisque si l'on note $X$ l'ensemble des monômes dominants des syzygies, si $X$ était régulier, alors son intersection avec le langage régulier $\{(1, 11, a^nb^me) \;|\; n, m \geq 1\}$ devrait être régulier, or cette intersection est le langage $\{(1, 11, a^nb^ne) \;|\; n \geq 1 \}$, qui n'est clairement pas régulier par le lemme de pompage.

Soit $n \geq 1$, commençons par montrer $(1, 11, a^nb^ne)$ est un monôme dominant de syzygie.
On vérifie que~:

\begin{align*}
(1, 1, a^{n-1}b^ne) & + \sum_{k = 2}^{n} (tc^{k - 2}, 7, a^{n - k}d^{k - 1}b^{n - k + 1}e) \\
& + \sum_{k = 1}^{n - 1} (tc^ka^{n - k}d^{k - 1}, 6, b^{n - k - 1}e) \\
& + (c^nd^{n - 1}, 8, 1) \\
& + (1, 11, a^nb^ne) \\
& + \sum_{k = 1}^n (tc^nd^{n - k}, 9, b^{k - 1}e) \\
& + \sum_{k = 1}^n (tc^{n - k}, 10, a^{k - 1}b^ne) \\
& + \sum_{k = 1}^{n} \sum_{l = 0}^{n - k - 1} (tc^{k}a^{l}, 2, a^{n - l - k - 1}d^{k - 1}b^{n - k + 1} e) \\
& + \sum_{k = 1}^{n} \sum_{l = 0}^{k - 2} (tc^{k}a^{n - k}d^l, 3, d^{k - 2 - l}b^{n - k - 1}e) \\
& + \sum_{k = 1}^{n - 1} \sum_{l = 0}^{k - 1} (tc^{k}a^{n - k}d^l, 5, d^{k - 1 - l}b^{n - k - 2}e) \\
& + \sum_{k = 1}^{n - 1} \sum_{l = 0}^{n - k - 1} (tc^{k}a^{l}, 4, a^{n - l - k - 1}d^{k}b^{n - k} e) \\
\end{align*}
est une syzygie de monôme dominant $(1, 11, a^nb^ne)$.

Soit $\sigma$ une syzygie de monôme dominant $(1, 11, a^{n}b^{m}e)$, et essayons de montrer que $n = m$.
Soit $G$ le graphe dont les sommets sont les éléments de $M$, et pour chaque $\alpha(l, i, r)$ terme de $\sigma$, si $g_i$ s'écrit $X - Y$, on crée un arc entre $lXr$ et $lYr$.
Alors, soit $W$ l'ensemble des sommets d'une des formes suivantes~:
\begin{align*}
& sa^nb^ne, \\
& tc^pa^qxa^rd^nb^me, \\
& tc^pa^qd^rxd^nb^me, \\
& tc^pa^qya^rd^nb^me, \\
& tc^pa^qd^ryd^nb^me, \\
& tc^pza^qb^re, \\
& tc^pd^qzb^re
\end{align*}
Chaque sommet de $W$ ne peut être relié qu'à des sommets de $W$, et chaque sommet de $W$ est relié à au plus deux sommets. Ainsi, la composante connexe de $sa^nb^ne$ dans $\sigma$ est une chaîne ou un cycle. Comme $\sigma$ est une syzygie, il doit s'agir d'un cycle, ce qui impose $n = m$.

\end{proof}

On pourrait remarquer que la suite de polynômes n'admet pas forcément de base de Gröbner finie.
Cependant, grâce à l'astuce suivante, on peut transformer l'exemple précédent pour qu'il ait une base de Gröbner finie~: on ajoute $\infty$ à la suite des polynômes, et pour toute lettre $\alpha$, on ajoute $\infty - \alpha$ à la suite des polynômes.

La nouvelle suite de polynômes ainsi contruite a une base de Gröbner finie ; elle contient pour tout symbole $\alpha$ le polynôme $\alpha$.
Soit $X$ l'ensemble des monômes dominants des syzygies avant la transformation, $Y$ l'ensemble des monômes dominants des syzygies après la transformation, et $Z$ le langage~:
$$\{(l, i, r) \in \M \times \N \times \M \;|\; l \LM(g_i) r \mbox{ ne contient pas } \infty\}$$
On a~:
$$X = Y \cap Z$$
Comme $Z$ est un langage régulier, si $Y$ était un langage régulier, alors $X$ le serait aussi.
Le langage $Y$ n'est donc pas régulier.

On a donc construit une suite de polynômes qui admet une base de Gröbner finie, mais dont l'ensemble des monômes dominants des syzygies n'est pas un langage régulier.

De plus, on peut rendre l'ordre isomorphe à $\N$ en choisissant deglex et en remplaçant le symbole $\infty$ par $\infty^{10}$ et $s$ par $s^5$, car les degrés varient d'au plus une constante lors des réécritures contenues dans une syzygie.

\subsubsection*{Langage algébrique ?}

La question naturelle suivante est de se demander s'il existe un automate à pile qui reconnaît les monômes dominants des syzygies.
Il se trouve que la technique de la section précédente se généralise~: on utilise le même genre de construction, en utilisant que le langage $\{a^nb^nc^n \;|\; n \in \N\}$ n'est pas algébrique.
On en déduit que les monômes dominants de syzygies ne sont pas toujours reconnaissables par un automate à pile. Comme dans le cas du langage régulier, on peut modifier l'exemple pour que l'ordre soit isomorphe à $\N$, car les degrés varient d'au plus une constante lors des réécritures contenues dans une syzygie.

\section*{Langage décidable}

Si l'ordre monomial est isomorphe pour les ordres à $\N$, il existe bien un algorithme pour décider si un triplet $(l, i, r)$ est un monôme dominant de syzygie, mais il s'avère qu'il est très peu efficace~:
On considère l'espace vectoriel $V$ engendré par les triplets plus petits que $(l, i, r)$, qui est de dimension finie.
Décider si $(l, i, r)$ est un monôme dominant de syzygie revient à décider s'il existe des éléments $\sigma$ de $V$ tel que le coefficient de $(l, i, r)$ dans $\sigma$ est $1$, et tel que $E(\sigma) = 0$, ce qui revient à décider si un système linéaire avec un nombre fini d'équations admet des solutions, ce qui est bien entendu décidable.

En revanche, si l'ordre monomial n'est pas isomorphe à $\N$, alors il n'existe pas forcément d'algorithme pour décider si un triplet $(l, i, r)$ est un monôme dominant de syzygie. Nous avons en effet trouvé une construction qui ramène ce problème au problème de l'arrêt.

\nocite{*}
\printbibliography

\end{document}
