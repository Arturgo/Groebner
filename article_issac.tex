\documentclass[sigconf]{acmart}

\copyrightyear{2021}
\acmYear{2021}
\setcopyright{rightsretained}
\acmConference[ISSAC '22]{International Symposium on Symbolic and Algebraic Computation}{July 4--7, 2022}{Lille, France}
\acmBooktitle{International Symposium on Symbolic and Algebraic Computation (ISSAC '22), July 4--7, 2022, Lille, France}
\acmPrice{15.00}
\acmDOI{XX.XXX/XXXXXX.XXXXXX}
\acmISBN{XXXXXXXXXXXXXXXXXXX}


\setlength{\paperheight}{11in}
\setlength{\paperwidth}{8.5in}

\usepackage[utf8]{inputenc}
\usepackage[T1]{fontenc}


%\usepackage{lmodern}
%\usepackage[a4paper]{geometry}

\usepackage{amsmath,  mathrsfs}
\usepackage[all]{xy}
\usepackage{stmaryrd}

\usepackage{tikz}
\usepackage{hyperref}


\usepackage{algorithm}
\usepackage[noend]{algorithmic}
\renewcommand{\algorithmicrequire}{\textbf{Input:}}
\renewcommand{\algorithmicensure}{\textbf{Output:}}
%\usepackage{graphicx}

\usepackage[english]{babel}
\usepackage{amsthm}
\theoremstyle{plain}
\newtheorem{lem}{Lemma}[section]
\newtheorem{prop}[lem]{Proposition}
\newtheorem{thm}[lem]{Theorem}
\newtheorem{cor}[lem]{Corollaire}
\newtheorem{axi}{Axiome}

\theoremstyle{definition}
\newtheorem{defn}[lem]{Definition}

\theoremstyle{remark}
\newtheorem{step}{\'Etape}
\newtheorem*{rmk}{Remark}
\newtheorem*{expl}{Example}

\newcommand{\N}{\mathbb{N}}
\newcommand{\NN}{\mathbb N}
\newcommand{\Z}{\mathbb{Z}}
\newcommand{\Zp}{\mathbb{Z}_p}
\newcommand{\Q}{\mathbb{Q}}
\newcommand{\Qp}{\Q_p}
\newcommand\cO{\mathcal{O}}
\newcommand{\softO}{O\tilde{~}}
\newcommand{\ddx}{\frac{\mathrm{d}\phantom{x}}{\mathrm{d}x}}
\newcommand{\OKun}{O_K([0,1])}
\newcommand{\val}{\mathrm{val}}


\newcommand\cY{\mathcal{Y}}
\newcommand{\ud}{\mathrm{d}}

\newcommand{\LT}{LT}
\newcommand{\LC}{LC}
\newcommand{\LM}{LM}

\newcommand\Zpt{\Zp\llbracket t \rrbracket}
\newcommand\Qpt{\Qp\llbracket t \rrbracket}
\newcommand\wrt{with respect to\xspace}

\newcommand{\X}{\mathbf{X}}
\newcommand{\Y}{\mathbf{Y}}
\renewcommand{\i}{\mathbf{i}}
\renewcommand{\j}{\mathbf{j}}
\renewcommand{\r}{\mathbf{r}}

\newcommand{\ifnonempty}[3]{%
  % IF #1 is empty THEN #3 ELSE #2
  \def\tempa{}%
  \def\tempb{#1}%
  \ifx\tempa\tempb % Empty case
  #3 
  \else            % Non-empty case
  #2
  \fi}

\newcommand{\Kz}{K^\circ}
\newcommand{\KzX}[1][]{K\{ \X \ifnonempty{#1}{; #1}{} \}^\circ}
\newcommand{\KX}[1][]{K \{ \X \ifnonempty{#1}{; #1}{} \}}
\newcommand{\Lz}{L^\circ}
\newcommand{\LzX}[1][]{L\{ \X \ifnonempty{#1}{; #1}{} \}^\circ}
\newcommand{\LX}[1][]{L \{ \X \ifnonempty{#1}{; #1}{} \}}
\newcommand{\TzX}[1][]{T\{ \X \ifnonempty{#1}{; #1}{} \}^\circ}
\newcommand{\TX}[1][]{T \{ \X \ifnonempty{#1}{; #1}{} \}}
\newcommand{\TT}{\mathbb T}
\newcommand{\TTzX}[1][]{\TT\{ \X \ifnonempty{#1}{; #1}{} \}^\circ}
\newcommand{\TTX}[1][]{\TT \{ \X \ifnonempty{#1}{; #1}{} \}}
\newcommand{\LKX}[1][]{\eta^{\NN} \KX[#1]}

\newcommand{\Kb}{\bar{K}}
\newcommand{\KbX}{\Kb \{ \X \}}

\newcommand\padic{$p$-adic\xspace}

\def\todo#1{\ \!\!{\color{red} #1}}


\def\eqdef{\stackrel{\text{def}}{=}}




\begin{document}

\fancyhead{}

\title{On The Difficulty of Computing Non-Commutative Signature Gröbner Bases}

\author{Cyrille Chenavier}
\affiliation{Universit\'e de Limoges;
  \institution{CNRS, XLIM UMR 7252}
  \city{Limoges}
  \country{France}  
  \postcode{87060}  
}
\email{cyrille.chenavier@unilim.fr}


\author{
  Arthur Léonard}
\affiliation{\'Ecole Normale Supérieure, 
  \institution{DI, DMA}
  \city{Paris}
  \country{France}  
}
\email{arthur.leonard@ens.psl.eu }

\author{
  Tristan Vaccon}
\affiliation{Universit\'e de Limoges;
  \institution{CNRS, XLIM UMR 7252}
  \city{Limoges}
  \country{France}  
  \postcode{87060}  
}
\email{tristan.vaccon@unilim.fr}


\thanks{This work was supported by...}


\begin{abstract}
NC GB, signature...\end{abstract}


\begin{CCSXML}
  <ccs2012>
  <concept>
  <concept_id>10010147.10010148.10010149.10010150</concept_id>
  <concept_desc>Computing methodologies~Algebraic algorithms</concept_desc>
  <concept_significance>500</concept_significance>
  </concept>
  </ccs2012>
\end{CCSXML}

\ccsdesc[500]{Computing methodologies~Algebraic algorithms}

% \vspace{-1mm}
% \ccsdesc[500]{Computing methodologies~Algebraic algorithms}
% \printccsdesc

\vspace{-1.5mm}
\terms{Algorithms, Theory}





\keywords{Algorithms, Gröbner bases, non-commutative algebra}

\maketitle

%\clubpenalty=10000
\widowpenalty = 10000
\addtolength{\textfloatsep}{-0.45cm} % Distance between float (e.g. [t]) and text

\section{Introduction}
Commutative part.
Modern algorithms:
\cite{F99,F5,FGLM}.
Survey \cite{EF17}
New take  \cite{G2V, GVW}
Applications \cite{LWXZ, CVV20, VVY21}

NC part.
Freaks \cite{MU00}.
Attempt \cite{K14}.
GVW for skew polynomial rings \cite{ZZ15}
Final complete
nc version \cite{HV21}.

\section{Notations}

In this article,
we work with
the notations defined
in \cite{HV21}.

%Aller jusqu'à la Def 18
%de l'article HV


\section{Rationality}

\section{Algebraicity}

\section{Finiteness}

\section{Implementation}

\bibliographystyle{plain}
\begin{thebibliography}{99}
  \renewcommand{\itemsep}{0em}
  % \renewcommand{\itemsep}{0.14em}

\bibitem{Bu65}
  Buchberger Bruno,
\newblock{{Ein Algorithmus zum Auffinden der Basiselemente des Restklassenringes nach einem nulldimensionalen Polynomideal (An Algorithm for Finding the Basis Elements in the Residue Class Ring Modulo a Zero Dimensional Polynomial Ideal)}},
\newblock English translation in J. of Symbolic Computation, Special Issue on Logic, Mathematics, and Computer Science: Interactions. Vol. 41, Number 3-4, Pages 475--511, 2006

  \bibitem{CVV20}
  Caruso, X., Vaccon T., Verron T.,
  \newblock {Signature-based algorithms for Gröbner bases over Tate algebras},
  \newblock {in Proceedings: ISSAC 2020, Kalamata, Greece.}

\bibitem{EF17}
Eder Christian and Faug{\`{e}}re Jean{-}Charles,
\newblock {A survey on signature-based algorithms for computing Gr{\"{o}}bner bases},
\newblock J. of Symbolic Computation, 2017


  \bibitem{F99}
  Faugère Jean-Charles,
  \newblock  {{A new efficient algorithm for computing Gröbner bases (F4)}},
  \newblock  {Journal of Pure and Applied Algebra}, {1999}

\bibitem{F5}
Faugère, Jean-Charles,
\newblock {A new efficient algorithm for computing Gröbner bases without reduction to zero (F5)},
\newblock{in Proceedings: {ISSAC'02}.}


  \bibitem{FGLM}
  Faugère, J.-C., Gianni, P., Lazard, D., Mora, T.,
  \newblock{Efficient computation of zero-dimensional Gr{\"o}bner bases by change of ordering},
  \newblock{J. of   Symbolic Computation 16~(4), 329--344, 1993}

  \bibitem{G2V}
Gao Shuhong, Guan Yinhua and Volny IV Frank,
\newblock{ A new incremental algorithm for computing Groebner bases},
\newblock{In Proceedings: {ISSAC'10}.}
%\newblock{ In Proceedings of the 2010 International Symposium on Symbolic and Algebraic Computation (ISSAC ’10). ACM, New York, NY, USA, 13–19.}


  \bibitem{GVW}
Gao Shuhong, Volny IV Frank, and Wang Mingsheng,
\newblock{ A new framework for computing Gröbner bases},
\newblock{ Mathematics of computation, 2016, vol. 85, no 297, p. 449-465.}

\bibitem{K14}
S.A. King,
\newblock {A non-commutative F5 algorithm with an application to the computation of Loewy layers},
\newblock {J. of Symbolic Computation} 65 (2014), pp. 111-129

  \bibitem{LWXZ}
  Lu Dong, Wang Dingkang, Xiao Fanghiu, Zhou Jie,
  \newblock  {{Extending the GVW Algorithm to Local Ring}},
  \newblock {Proceedings of 43th International Symposium on Symbolic and Algebraic Computation, {ISSAC}'18, New York, USA}



\bibitem{MU00}
E.Green, T. Mora, V.Ufnarovski,
\newblock The Non-Commutative Gröbner Freaks.
\newblock Symbolic Rewriting Techniques. Vol. 15. 



\bibitem{HV21}
C. Hofstadler, T. Verron,
\newblock {
Signature Gröbner bases, bases of syzygies and cofactor reconstruction in the free algebra}, 
\newblock  	arXiv:2107.14675, 2021




  \bibitem{Sage}
  \newblock {{S}ageMath, the {S}age {M}athematics {S}oftware {S}ystem ({V}ersion 9.2)}, The Sage Development Team, 2020, \url{http://www.sagemath.org}

\bibitem{VVY21}
T.Vaccon, T.Verron, K.Yokoyama,
\newblock
{On affine tropical F5 algorithms},
\newblock {
Journal of Symbolic Computation},
Volume 102,
2021.

\bibitem{ZZ15}
X. Zhao, Y. Zhang,
\newblock {A signature-based algorithm for computing Gröbner-Shirshov bases in skew solvable polynomial rings}
\newblock Open Mathematics, Volume: 13, Issue: 1, 2015. 

\end{thebibliography}

\end{document}

