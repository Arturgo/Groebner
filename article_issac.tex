\documentclass[sigconf]{acmart}

\copyrightyear{2021}
\acmYear{2021}
\setcopyright{rightsretained}
\acmConference[ISSAC '22]{International Symposium on Symbolic and Algebraic Computation}{July 4--7, 2022}{Lille, France}
\acmBooktitle{International Symposium on Symbolic and Algebraic Computation (ISSAC '22), July 4--7, 2022, Lille, France}
\acmPrice{15.00}
\acmDOI{XX.XXX/XXXXXX.XXXXXX}
\acmISBN{XXXXXXXXXXXXXXXXXXX}


\setlength{\paperheight}{11in}
\setlength{\paperwidth}{8.5in}

\usepackage[utf8]{inputenc}
\usepackage[T1]{fontenc}


%\usepackage{lmodern}
%\usepackage[a4paper]{geometry}

\usepackage{amsmath,  mathrsfs}
\usepackage[all]{xy}
\usepackage{stmaryrd}

\usepackage{tikz}
\usepackage{hyperref}


\usepackage{algorithm}
\usepackage[noend]{algorithmic}
\renewcommand{\algorithmicrequire}{\textbf{Input:}}
\renewcommand{\algorithmicensure}{\textbf{Output:}}
%\usepackage{graphicx}

\usepackage[english]{babel}
\usepackage{amsthm}
\theoremstyle{plain}
\newtheorem{lem}{Lemma}[section]
\newtheorem{prop}[lem]{Proposition}
\newtheorem{thm}[lem]{Theorem}
\newtheorem{cor}[lem]{Corollaire}
\newtheorem{axi}{Axiome}

\theoremstyle{definition}
\newtheorem{defn}[lem]{Definition}

\theoremstyle{remark}
\newtheorem{step}{\'Etape}
\newtheorem*{rmk}{Remark}
\newtheorem*{expl}{Example}

\newcommand{\N}{\mathbb{N}}
\newcommand{\NN}{\mathbb N}
\newcommand{\MM}{\mathbb M}

\newcommand{\Z}{\mathbb{Z}}
\newcommand{\Zp}{\mathbb{Z}_p}
\newcommand{\Q}{\mathbb{Q}}
\newcommand{\Qp}{\Q_p}
\newcommand\cO{\mathcal{O}}
\newcommand{\softO}{O\tilde{~}}
\newcommand{\ddx}{\frac{\mathrm{d}\phantom{x}}{\mathrm{d}x}}
\newcommand{\OKun}{O_K([0,1])}
\newcommand{\val}{\mathrm{val}}


\newcommand\cY{\mathcal{Y}}
\newcommand{\ud}{\mathrm{d}}

\newcommand{\LT}{LT}
\newcommand{\LC}{LC}
\newcommand{\LM}{LM}

\newcommand\Zpt{\Zp\llbracket t \rrbracket}
\newcommand\Qpt{\Qp\llbracket t \rrbracket}
\newcommand\wrt{with respect to\xspace}

\newcommand{\X}{\mathbf{X}}
\newcommand{\Y}{\mathbf{Y}}
\renewcommand{\i}{\mathbf{i}}
\renewcommand{\j}{\mathbf{j}}
\renewcommand{\r}{\mathbf{r}}

\newcommand{\ifnonempty}[3]{%
  % IF #1 is empty THEN #3 ELSE #2
  \def\tempa{}%
  \def\tempb{#1}%
  \ifx\tempa\tempb % Empty case
  #3 
  \else            % Non-empty case
  #2
  \fi}

\newcommand{\KX}{K \left\langle X \right\rangle}


\def\todo#1{\ \!\!{\color{red} #1}}


\def\eqdef{\stackrel{\text{def}}{=}}




\begin{document}

\fancyhead{}

\title{On The Difficulty of Computing Non-Commutative Signature Gröbner Bases}

\author{Cyrille Chenavier}
\affiliation{Universit\'e de Limoges;
  \institution{CNRS, XLIM UMR 7252}
  \city{Limoges}
  \country{France}  
  \postcode{87060}  
}
\email{cyrille.chenavier@unilim.fr}


\author{
  Arthur Léonard}
\affiliation{\'Ecole Normale Supérieure, 
  \institution{DI, DMA}
  \city{Paris}
  \country{France}  
}
\email{arthur.leonard@ens.psl.eu }

\author{
  Tristan Vaccon}
\affiliation{Universit\'e de Limoges;
  \institution{CNRS, XLIM UMR 7252}
  \city{Limoges}
  \country{France}  
  \postcode{87060}  
}
\email{tristan.vaccon@unilim.fr}


\thanks{This work was supported by...}


\begin{abstract}
NC GB, signature...\end{abstract}


\begin{CCSXML}
  <ccs2012>
  <concept>
  <concept_id>10010147.10010148.10010149.10010150</concept_id>
  <concept_desc>Computing methodologies~Algebraic algorithms</concept_desc>
  <concept_significance>500</concept_significance>
  </concept>
  </ccs2012>
\end{CCSXML}

\ccsdesc[500]{Computing methodologies~Algebraic algorithms}

% \vspace{-1mm}
% \ccsdesc[500]{Computing methodologies~Algebraic algorithms}
% \printccsdesc

\vspace{-1.5mm}
\terms{Algorithms, Theory}





\keywords{Algorithms, Gröbner bases, non-commutative algebra}

\maketitle

%\clubpenalty=10000
\widowpenalty = 10000
\addtolength{\textfloatsep}{-0.45cm} % Distance between float (e.g. [t]) and text

\section{Introduction}
Commutative part.
Modern algorithms:
\cite{F99,F5,FGLM}.
Survey \cite{EF17}
New take  \cite{G2V, GVW}
Applications \cite{LWXZ, CVV20, VVY21}

NC part.
Intro \cite{Mora94}
Freaks \cite{GMU00}.
F5 for right-modules over path algebra quotients \cite{K14}.
GVW for skew polynomial rings \cite{ZZ15}
Final complete
nc version \cite{HV21}.

\section{Notations}

In this article,
we work with
the notations defined
in \cite{HV21}.

%Aller jusqu'à la Def 18
%de l'article HV


Let $X= \{ x_1,\dots,x_n \}$
be a set of indeterminates.
We denote by $\left\langle X \right\rangle$ the
free monoid of words 
over $X$ and, for a field
$K$, we denote by
$\KX$ the
free algebra generated by $X$
over $K.$
We consider the elements
in $\KX$
as noncommutative polynomials
with coefficients in $K$
and indeterminates in $X$
(indeterminates commute
with coefficientes but not
with each other).

If $F \subset \KX,$
we denote by $(F)$
the two-sided ideal 
generated by $F$:
\[(F) = \left\lbrace \sum_{i=1}^d a_i f_i b_i \: \mid \: f_i \in F, \: a_i,b_i \in \KX, \: d \in \NN  \right\rbrace. \]

Let $\leq$ be a monomial 
ordering on $\left\langle X \right\rangle$.
For $f \in \KX \setminus \{ 0 \}$, we denote by $\LC (f), \LT (f), \LM (f)$
the leading coefficients,
leading terms and 
leading monomials of $f$
(with $\LT(f) = \LC(f) \LM(f)$).

We recall that $G \subset (F)$
is a Gröbner basis
of $(F)$ if
$\LM (G)$ generates $LM\left( (F) \right)$ as
monoids inside $\left\langle X \right\rangle.$

For $r \in \NN$ we denote
by $\mathscr{F}_r = \left( \KX \otimes \KX \right)^r$
the free $\KX$-bimodule
of rank $r$, with canonical
basis $\varepsilon_1,\dots,\varepsilon_r$
where $\varepsilon_i =(0,\dots,0,1 \otimes 1,0,\dots,0)$ ($1 \otimes 1$ in the $i$-th position) for $i \in \llbracket 1,r \rrbracket.$
Let $\MM \left( \mathscr{F}_r \right)= \left\lbrace a \varepsilon_i b \mid a,b \in \left\langle X \right\rangle, \: i \in \llbracket 1,r\rrbracket \right\rbrace$
be the set of module monomials in $\mathscr{F}_r.$ It is a $K$-vector space basis
of $\mathscr{F}_r.$
Let $\leq_\MM$ be a module
ordering on $\MM \left( \mathscr{F}_r \right).$
We recall that it means that
$\leq_\MM$ is a well-ordering
compatible with 
scalar multiplication, that
is 
$\mu \leq_\MM \mu'$ implies
$a \mu b \leq_\MM a \mu' b$
for all $\mu,\mu' \in \MM \left( \mathscr{F}_r \right)$
and $a, b \in \left\langle X \right\rangle.$
We assume that 
$\leq$ and $\leq_\MM$
are compatible
(\textit{i.e.} for any
$a,b \in \left\langle X \right\rangle$, $i \in \llbracket 1,r \rrbracket$,
$a \leq b \Leftrightarrow a \varepsilon_i \leq_\MM b \varepsilon_i \Leftrightarrow \varepsilon_i a \leq_\MM \varepsilon_i b$).
We also assume that 
$\leq_\MM$ is fair (for any
$\mu \in \MM (\mathscr{F}_r)$,
the set $\left\lbrace \mu' \in \MM (\mathscr{F}_r) \mid \mu' \leq_\MM \mu \right\rbrace$ is finite).

If $\alpha \in \mathscr{F}_r$, it can be uniquely
written in the basis 
$\MM \left( \mathscr{F}_r \right) $
and the biggest monomial
involved is its signature
$\mathfrak{s}(\alpha) \in
\mathscr{F}_r$. 

If $M \subset \mathscr{F}_r$
is a $\KX$-submodule,
then $G \subset M$ is called
a Gröbner basis of $M$ if
\[\mathfrak{s}(M)= \left\lbrace a \mathfrak{s}(\gamma) b \mid a,b \in \left\langle X \right\rangle, \gamma \in G \right\rbrace. \]

Let $f_1,\dots,f_r \in \KX$
be some polynomials
generating the ideal $I=(f_1,\dots,f_r).$
They define a 
$\KX$-module homomorphism:

\[
\begin{array}{ccccc}
&& \mathscr{F}_r & \rightarrow & \KX \\
\overline{\cdot}& : &\alpha = \sum_i c_i a_i \varepsilon_{j_i} b_i&\mapsto& \overline{\alpha}=\sum_i c_i a_i f_{j_i} b_i. \\
\end{array}\]

Elements $(f,\alpha) \in \KX \times \mathscr{F}_r$
such that $f=\overline{\alpha}$
are called signature polynomials.
We denote by $f^{[\alpha]}$
such a pair. 
For such a pair, necessarily,
$f \in I.$
We denote by $I^{[\Sigma ]}=\left\lbrace f^{[\alpha]} \mid f=\overline{\alpha} \right\rbrace \subset I \times \mathscr{F}_r$
the set of all signature
polynomials.

For $f^{[\alpha]},g^{[\beta]},
h^{[\gamma]} \in I^{\Sigma},$
with $\alpha, g \neq 0,$
we say that 
$f^{[\alpha]}$ $\mathfrak{s}$-reduces to
$g^{[\beta]}$ by
$h^{[\gamma]}$
if there exist $a,b \in 
\left\langle X \right\rangle$
such that:
\begin{itemize}
\item $a \LM(h)b \in supp(f),$
\item $\mathfrak{s}(a \gamma b) \leq \mathfrak{s}(\alpha),$ and
\item $g^{[\beta]}=f^{[\alpha]}-\frac{coeff(f,a \LM(h)b}{\LC(g)} a h^{[\gamma]}b.$
\end{itemize}
Reduction by a set $G^{[\Sigma]} \in I^{[\Sigma]}$
is then defined as the 
reflexive, transitive closure
of the reduction by elements
of $G^{[\Sigma]} $.

Finally, $G^{[\Sigma]} $
is called a \textit{signature
Gröbner basis} of $I^{[\Sigma]}$ if all
$f^{[\alpha]} \in I^{[\Sigma]}$ $\mathfrak{s}$-reduce to zero by $G^{[\Sigma]}.$

\section{Rationality}

\section{Algebraicity}

\section{Finiteness}

\section{Implementation}

\bibliographystyle{plain}
\begin{thebibliography}{99}
  \renewcommand{\itemsep}{0em}
  % \renewcommand{\itemsep}{0.14em}

\bibitem{Bu65}
  Buchberger Bruno,
\newblock{{Ein Algorithmus zum Auffinden der Basiselemente des Restklassenringes nach einem nulldimensionalen Polynomideal (An Algorithm for Finding the Basis Elements in the Residue Class Ring Modulo a Zero Dimensional Polynomial Ideal)}},
\newblock English translation in J. of Symbolic Computation, Special Issue on Logic, Mathematics, and Computer Science: Interactions. Vol. 41, Number 3-4, Pages 475--511, 2006

  \bibitem{CVV20}
  Caruso, X., Vaccon T., Verron T.,
  \newblock {Signature-based algorithms for Gröbner bases over Tate algebras},
  \newblock {in Proceedings: ISSAC 2020, Kalamata, Greece.}

\bibitem{EF17}
Eder Christian and Faug{\`{e}}re Jean{-}Charles,
\newblock {A survey on signature-based algorithms for computing Gr{\"{o}}bner bases},
\newblock J. of Symbolic Computation, 2017


  \bibitem{F99}
  Faugère Jean-Charles,
  \newblock  {{A new efficient algorithm for computing Gröbner bases (F4)}},
  \newblock  {Journal of Pure and Applied Algebra}, {1999}

\bibitem{F5}
Faugère, Jean-Charles,
\newblock {A new efficient algorithm for computing Gröbner bases without reduction to zero (F5)},
\newblock{in Proceedings: {ISSAC'02}.}


  \bibitem{FGLM}
  Faugère, J.-C., Gianni, P., Lazard, D., Mora, T.,
  \newblock{Efficient computation of zero-dimensional Gr{\"o}bner bases by change of ordering},
  \newblock{J. of   Symbolic Computation 16~(4), 329--344, 1993}

  \bibitem{G2V}
Gao Shuhong, Guan Yinhua and Volny IV Frank,
\newblock{ A new incremental algorithm for computing Groebner bases},
\newblock{In Proceedings: {ISSAC'10}.}
%\newblock{ In Proceedings of the 2010 International Symposium on Symbolic and Algebraic Computation (ISSAC ’10). ACM, New York, NY, USA, 13–19.}


  \bibitem{GVW}
Gao Shuhong, Volny IV Frank, and Wang Mingsheng,
\newblock{ A new framework for computing Gröbner bases},
\newblock{ Mathematics of computation, 2016, vol. 85, no 297, p. 449-465.}

\bibitem{K14}
S.A. King,
\newblock {A non-commutative F5 algorithm with an application to the computation of Loewy layers},
\newblock {J. of Symbolic Computation} 65 (2014), pp. 111-129

  \bibitem{LWXZ}
  Lu Dong, Wang Dingkang, Xiao Fanghiu, Zhou Jie,
  \newblock  {{Extending the GVW Algorithm to Local Ring}},
  \newblock {Proceedings of 43th International Symposium on Symbolic and Algebraic Computation, {ISSAC}'18, New York, USA}



\bibitem{GMU00}
E.Green, T. Mora, V.Ufnarovski,
\newblock The Non-Commutative Gröbner Freaks.
\newblock Symbolic Rewriting Techniques. Vol. 15. 



\bibitem{HV21}
C. Hofstadler, T. Verron,
\newblock {
Signature Gröbner bases, bases of syzygies and cofactor reconstruction in the free algebra}, 
\newblock  	arXiv:2107.14675, 2021

\bibitem{Mora94}
T. Mora,
\newblock {
An  introduction to  commutative 
and noncommutative Gröbner 
bases },
\newblock Theoretical Computer Science 134, 1994. 


  \bibitem{Sage}
  \newblock {{S}ageMath, the {S}age {M}athematics {S}oftware {S}ystem ({V}ersion 9.2)}, The Sage Development Team, 2020, \url{http://www.sagemath.org}

\bibitem{VVY21}
T.Vaccon, T.Verron, K.Yokoyama,
\newblock
{On affine tropical F5 algorithms},
\newblock {
Journal of Symbolic Computation},
Volume 102,
2021.

\bibitem{ZZ15}
X. Zhao, Y. Zhang,
\newblock {A signature-based algorithm for computing Gröbner-Shirshov bases in skew solvable polynomial rings}
\newblock Open Mathematics, Volume: 13, Issue: 1, 2015. 

\end{thebibliography}

\end{document}

